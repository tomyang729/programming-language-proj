\section{Overview}

aaa

\section{Value/Importance/Impact}

aaa

\section{Similar work}

aaa

\subsection{Alda}

aaa

\subsection{Chuck}

aaa

\subsection{Overtone}

Overtone is an Open Source toolkit for designing synthesizers and collaborating with music. It provides\cite{Aaron16}:
\begin{itemize}
\item A Clojure API to the SuperCollider synthesis engine
\item A growing library of musical functions (scales, chords, rhythms, arpeggiators, etc.)
\item Metronome and timing system to support live-programming and sequencing
\item Plug and play MIDI device I/O
\item A full Open Sound Control (OSC) client and server implementation.
\item Pre-cache - a system for locally caching external assets such as .wav files
\end{itemize}

Overtone's idea is about sound generation. "Let me answer from the synthesis perspective - which is one of my main interests. Learning to design new synthesisers is a pretty dark art, and most of the books/resources I found take a very theory-centric stance which I found to not be particularly useful." Said the author of Overtone, Sam Aaron.\cite{Aaron13} The current design of ABCD is just let it compile to ABC, and let ABC deals with the sound generation part. The ultimate version of ABCD can be directly interpreted to play sound, so how Overtone hooks up with SuperCollider can be case-studied for our own implementation.

Another advantage of overtone is about collaborative programming. Overtone provides an API for querying and fetching sounds from http://freesopund.org and a global concurrent event stream\cite{Aaron16}. Sharing music is a huge part of enable people to write music. ABC is a great source in text format to share music, and Overtone is a great source to share music in computer programs. However, to understand Overtone program is not as straight forward as reading ABC notation. ABCD will have programming feature, so we enable both sides --- for people who writes complicated music programs and for people who just write plain ABC with some syntactic sugar to make their life easier.

From a language design perspective, Overtones design is based on Clojure. Two core component of Overtone are synths and ugens(unit-generator). Synths are trees of ugens. Ugens are standard Clojure functions and return data-structures which are understood by the macros demo and defsynth. You can pass arguments to the ugen functions to specify their behaviour. Synths are not ugens. Calling a ugen function returns a data structure which can be used in a synth design. Calling a synth as a function triggers (i.e. plays) that synth.\cite{Aaron14} The story of overtone gives us some hints on the language design --- we probably need to define what are the automics in the music world and make them first-class members of our language. However, for a sequence of sounds, they are probably playable/translatable objects.


\section{Potential Project}

\subsection{Features}
\subsection{Concrete Syntax}
\subsection{Parse into AST}

After some research, we have found a few implmentations of parsing ABC. Sergi Mansilla has implemented an ABC parser in JavaScript, which parses the ABC notation to a JSON object\cite{Mansilla12}. Remo Dentato has developed a C library to parse the ABC notation\cite{Dentato09}, however, the source code can no longer be found on the website and the usage description is quite complicated. However, on Dentato's website, it offers some high-level design graph of the scanner and documentation on tokens. Hans H\"{o}glund has implemented a parser in Haskell\cite{Hoglund15}. H\"{o}glund's implmentation has a few limitations such as do not support volatile features, text strings, and macros. abc4j is a Java library that provides API to handle ABC music notation using Java. The source code can be downloaded on \url{https://code.google.com/archive/p/abc4j/source}. However, it only supports the v1.6 standard of ABC (the current version is v2.1). There are many other people who tried to implment parser for ABC but failed or stopped their project.

\subsection{Compiler}

\section{Conclusions}

This paragraph will end the body of this sample document.
