\section{Introduction}

The \textit{proceedings} are the records of a conference.\footnote{This
  is a footnote}  ACM seeks
to give these conference by-products a uniform, high-quality
appearance.  To do this, ACM has some rigid requirements for the
format of the proceedings documents: there is a specified format
(balanced double columns), a specified set of fonts (Arial or
Helvetica and Times Roman) in certain specified sizes, a specified
live area, centered on the page, specified size of margins, specified
column width and gutter size.

\section{How it fulfills the project type requirements}

there needs to be some person (your user, who cannot just be you) who might realistically want to use your 100\% level to solve problems in their domain over the existing choices available
Ex a trumpet player may want a program that transposes songs for C instruments to scores for B flat instruments
A composer writing a score in ABC may wish to change the key of the last x bars of their song
An arranger may wish to add a harmony to every instance of a certain note
A producer may wish to generate part of a song or a whole song programmatically
All of these can be accomplished as programs written in ABCD. ABC alone is not powerful enough to do these programmatically (CONFIRM THIS) -- it would have to be done manually. This could be done using another programming language, but  ABC parsing and writing would need to be implemented, where ABCD would support those natively.

\section{How it could be a full 100\% project}
We intend to design a Domain-Specific Language extending the ABC music notation language. ABC is similar in power and purpose to LaTeX: it allows a user to hand-type a document that represents a piece of sheet music, and can be compiled to a PDF or played aloud using appropriate software. Our language, ABCD, will enhance ABC with common programming features such as identifiers and functions, empowering users to programmatically modify existing ABC documents and generate new music. 
	A full implementation of the language would involve:

An EBNF defining concrete syntax found in ABCD that are not in ABC (and possibly including the concrete syntax for ABC)
An ABCD parser
An abstract syntax implementation
Evaluation implementation:
A compiler to produce ABC files from the ABCD abstract syntax OR
An interpreter to run ABCD programs, which have ABC files as input and output [deprecated]

	ABCD should have native support for using ABC files as input to ABCD programs, allowing users to easily modify existing ABC scores. There are two general ways we can imagine using ABCD: writing scripts that compile into ABC files after modification or creation, or writing programs that may accept ABC files as input and output modified or generated ABC files. By outputting ABC files, our users can leverage existing software that works with ABC, including music players and pdf compilers.
	For the scripting approach, an ABCD file may contain definitions of identifiers and functions as well as ABC notation. When compiled/evaluated, functions will modify, add, or remove ABC code in the script, and the script will evaluate to one or more ABC documents.
	For the programming approach, an ABCD file may contain definitions of identifiers and functions, as well as a main program. The program can optionally take one or more ABC files (or something else) as input, create and modify ABC code during its evaluation, and optionally output one or more ABC files. We may wish to consider modifying files in-place as well. It is worth noting that this form of ABCD program is not a document containing a ?tune? like an ABC file does, and trying to compile it to an ABC document is not meaningful. We can think of the AST as the final, target form of an ABCD program, where a script would ultimately be evaluated to an ABC file. We can think of an ABCD program as a composer that creates new tunes or as an editor or remixer who transforms existing tunes. From a programmer?s lens, we can view an ABCD script as akin to a LaTeX document or HTML file, and an ABCD program more like a C program we might write for the unix command line, a pipeline that performs a transformation on data files or strings.
	We would likely write our parser interpreter/compiler in python to harness its string processing capabilities, as ABC has a plain-text format where characters often need to be considered individually.
	For a final project, we would likely also write a small ABCD program to demonstrate a possible use case for ABCD and showcase some of its capabilities. This would likely be a program that applies some transposition, addition of harmonies to certain notes, and some rhythm changes, and we would demonstrate by applying the program to a small ABC tune and playing the result or printing it as a pdf.
 
\section{List of possibly useful links}
The following is a list of sources we would consult during the project. 

First of all, http://abcnotation.com is a general site for all ABC related documents. In particular, http://abcnotation.com/examples contains straightforward specifications (illustrated by examples) for how ABC works as a markup language that represents sheet music. http://abcnotation.com/software provides a list of command lines and software packages that can read or edit ABC. 

Another source that allows us to understand ABC better and, more importantly, implement ABC software applications is the ABC wiki page: http://abcnotation.com/wiki/abc:standard:v2.2 In addition to the formal specifications of ABC, this site also provides information about ABC extensions, file readers, transposition operations, transposition macros, etc. 

http://www.mandolintab.net/abcconverter.php is an online converter that we can use for converting ABC into sheet music and export the file as MIDI, PDF, etc. This could be useful for testing whether a ABC file is valid. We may also use EasyABC (http://www.nilsliberg.se/ksp/easyabc/), an open source ABC editor, if we find the converter no sufficient

We can also read about the limitations of ABC at https://music.stackexchange.com/questions/23841/what-are-the-limitations-of-the-abc-notation-format
, where many ABC users and software developers posted their experiences about what are not supported in ABC.



\section{Conclusions}
This paragraph will end the body of this sample document.
Remember that you might still have Acknowledgments or
Appendices; brief samples of these
follow.  There is still the Bibliography to deal with; and
we will make a disclaimer about that here: with the exception
of the reference to the \LaTeX\ book, the citations in
this paper are to articles which have nothing to
do with the present subject and are used as
examples only.
%\end{document}  % This is where a 'short' article might terminate


\begin{acks}
  The authors would like to thank Dr. Yuhua Li for providing the
  MATLAB code of the \textit{BEPS} method.

  The authors would also like to thank the anonymous referees for
  their valuable comments and helpful suggestions. The work is
  supported by the \grantsponsor{GS501100001809}{National Natural
    Science Foundation of
    China}{http://dx.doi.org/10.13039/501100001809} under Grant
  No.:~\grantnum{GS501100001809}{61273304}
  and~\grantnum[http://www.nnsf.cn/youngscientists]{GS501100001809}{Young
    Scientists' Support Program}.

\end{acks}
