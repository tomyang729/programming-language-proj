\section{List of starting point documents}
The following is a list of sources we would consult during the project. 

First of all, \url{http://abcnotation.com} is a general site for all ABC related documents. In particular, \url{http://abcnotation.com/examples} contains straightforward specifications (illustrated by examples) for how ABC works as a markup language that represents sheet music. \url{http://abcnotation.com/software} provides a list of command lines and software packages that can read or edit ABC. 

Another source that allows us to understand ABC better and, more importantly, implement ABC software applications is the ABC wiki page: \url{http://abcnotation.com/wiki/abc:standard:v2.2} In addition to the formal specifications of ABC, this site also provides information about ABC extensions, file readers, transposition operations, transposition macros, etc. It allows us to see the various features already supported by ABC, such as transposition. We would need to think about what other features we can implement and/or how to improve the existing features in our language.

\url{http://www.mandolintab.net/abcconverter.php} is an online converter that we can use for converting ABC into sheet music and export the file as MIDI, PDF, etc. It could be useful for testing whether a ABC file is valid. We may also use EasyABC (\url{http://www.nilsliberg.se/ksp/easyabc/}), an open source ABC editor, if we find the converter not sufficient.

We may also learn about the limitations of ABC at \url{https://music.stackexchange.com/questions/23841}, where many ABC users and software developers share their experiences of working with ABC, such as in parsing and in practical use.
