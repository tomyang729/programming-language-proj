\section{List of possibly useful links}
The following is a list of sources we would consult during the project. 

First of all, http://abcnotation.com is a general site for all ABC related documents. In particular, http://abcnotation.com/examples contains straightforward specifications (illustrated by examples) for how ABC works as a markup language that represents sheet music. http://abcnotation.com/software provides a list of command lines and software packages that can read or edit ABC. 

Another source that allows us to understand ABC better and, more importantly, implement ABC software applications is the ABC wiki page: http://abcnotation.com/wiki/abc:standard:v2.2 In addition to the formal specifications of ABC, this site also provides information about ABC extensions, file readers, transposition operations, transposition macros, etc. 

http://www.mandolintab.net/abcconverter.php is an online converter that we can use for converting ABC into sheet music and export the file as MIDI, PDF, etc. This could be useful for testing whether a ABC file is valid. We may also use EasyABC (http://www.nilsliberg.se/ksp/easyabc/), an open source ABC editor, if we find the converter no sufficient

We can also read about the limitations of ABC at https://music.stackexchange.com/questions/23841/what-are-the-limitations-of-the-abc-notation-format
, where many ABC users and software developers posted their experiences about what are not supported in ABC.